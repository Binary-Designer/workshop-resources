\documentclass{article}
\usepackage[sfdefault]{roboto}

\title{Working with Actuators in Embedded Systems}
\author{UQMARS}
\date{\today}

\begin{document}
\maketitle
\pagebreak

\section*{What is an Actuator?}
Just as a sensor is a component that 'senses', an actuator is a component that 'acts'. Essentially, an actuator is something you use to create motion from your system.
\subsection*{Types of Actuators}
There are two categories of actuators that you may deal with:
\begin{itemize}
    \item linear
    \item rotary
\end{itemize}
As their names suggest, these create linear motion (along a straight path) and rotational motion respectively.

\section*{Rotary Actuators}
Likely the type of actuator that you will initially have greater exposure to, a rotary actuator is responsible for making things spin. The standard types that you will be exposed to within embedded systems are:
\begin{itemize}
    \item Servo Motor
    \item Brushed DC Motor
    \item Brushless DC Motor
    \item Stepper Motor
\end{itemize}
% Expand on use case of each, etc.

\subsection*{Rotary Actuators Activity}
% Have the practical component of the linear actuators section here
% Likely just using DC Hobby Motors

\section*{Linear Actuators}
Linear actuators are useful when you seek axial extension. Standard varieties that you may encounter are:
\begin{itemize}
    \item Screw Actuators
    \item Wheel and Axle
    \item Hydraulic / Pneumatic Cylinders
\end{itemize}
% Expand

\subsection*{Linear Actuators Activity}
% Very short, probably assemble a rack-and-pinion or something.


\end{document}
