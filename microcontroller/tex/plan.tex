\documentclass[11pt]{article}

\title{Microcontroller Basics in C/C++}
\author{UQMARS}
\date{\today}

\begin{document}
\maketitle
\pagebreak
\section*{Microcontroller options}
ATmega328P - Can be used for courses such as ENGG1100\\ (I think it was banned for METR2800 and METR4810?? just the 328P though)
STM32 - Used in CSSE3010, ENGG2800\\
???? (Any others you can think of??)
\section*{How to start programing Microcontroller overview of Arduino IDE}
Pretty basic I think we can just provide some sort of resource they can look at in their own time
\section*{How to program Microcontroller with C}
\subsection*{Header files}
Microcontroller dependent headerfiles, need to emphasise importance of reading documentation properly
as each microcontroller can have small variations.
\subsection*{Makefile}
Having a makefile is crucial allowing us to use a cross-compiler to compile code onto chip.
Documentation may provide a template makefile
\subsection*{Simple Blink LED}
Equivalent of Hello World! in the embedded world. Teaches how to output to certain PINS.
\subsection*{Button Input to turn on LED}
Teaches students how to handle input.
\subsection*{Interrupt}
Teaches students not to use busy waits as it can chew up a lot of processor cycles for no reason.
\subsection*{PWM}
PWM for things such as motors and SG90 servos
\end{document}
