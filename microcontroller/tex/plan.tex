\documentclass[11pt]{article}
\usepackage{xcolor}

\title{Microcontroller Basics} % in C/C++}
\author{UQMARS}
\date{\today}

\begin{document}
\maketitle
\pagebreak
\section*{Microcontroller Developement Board options}
\begin{itemize}
    \item Arduino Family
    \begin{itemize}
        \item Arduino Uno (ATmega328P)
        \item Arduino Mega (ATmega2560)
        \item Ardiono Nano (ATmega328P)
    \end{itemize}
    \item Raspberry Pi Family 
    \begin{itemize}
        \item Raspberry Pi Model B
        \item Raspberry Pi Zero
        \item Raspberry Pi Pico
    \end{itemize}
    \item Espressif Family
    \begin{itemize}
        \item ESP8266 Development Boards
        \item ESP32 Development Boards
    \end{itemize}
\end{itemize}

\textcolor{red}{ATmega328P - Can be used for courses such as ENGG1100\\ (I think it was banned for METR2800 and METR4810?? just the 328P though)
STM32 - Used in CSSE3010, ENGG2800\\
???? (Any others you can think of?)}



\section*{Basic Microcontroller Programming - Intro to the Arduino IDE}

\textcolor{red}{Pretty basic I think we can just provide some
sort of resource they can look at in their own time}

\subsection*{Simple Blink LED}
Equivalent of Hello World! in the embedded world. Teaches how to output to certain PINS.
\subsection*{Button Input to turn on LED}
Teaches students how to handle input.
\subsection*{PWM}
PWM for things such as motors and SG90 servos.


\section*{Advanced Lessons}

\subsection*{Wireless Communication}
Wireless communication is essential for remotely operated devices/robotics. 
Although not required, you are usually encouraged to use wireless 
communication in the ENGG1100 and METR2800 group projects.
% Ill need a fact check here cause I didn't do METR2800
Alternatively, many groups use a long USB cable to remotely operate their 
machine - yuck. After this tutorial you will hopefully gain the confidence to blah blah....

\section*{How to program Microcontroller with C}
\subsection*{Header files}
Microcontroller dependent headerfiles, need to emphasise importance of reading documentation properly
as each microcontroller can have small variations.
\subsection*{Makefile}
Having a makefile is crucial allowing us to use a cross-compiler to compile code onto chip.
Documentation may provide a template makefile

\subsection*{Interrupt}
Teaches students not to use busy waits as it can chew up a lot of processor cycles for no reason.
\end{document}
